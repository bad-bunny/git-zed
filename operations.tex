\section{Operations}

\subsection{Add}

The add operation updates the index using the current content found in the
working tree, to prepare the content staged for the next commit. This command
can be performed multiple times before a commit. It only adds the content of the
specified file at the time the operation is run.

\begin{schema}{GitAddOk}
  \Delta Repository \\
  file?: Tree \\
  path?: Tree \\
  response!: GitResponse
\where
  index' = treeWithFile~(index, path?, file?) \\
  head' = head \\
  workingCopy' = workingCopy \\
  branch' = branch \\
  tag' = tag \\
  parent' = parent \\
  response! = Ok
\end{schema}

If the input combination of $file?$ and $path?$ is not part of the index tree,
then the file cannot be added:

\begin{schema}{GitAddCannotAdd}
  \Xi Repository \\
  file?: Tree \\
  path?: Tree \\
  response!: GitResponse
\where
  (index, path?, file?) \notin \dom treeWithFile \\
  response! = CannotAdd
\end{schema}

Finally, we define \texttt{GitAdd} as the disjunction of the previous two
schemas:

\begin{zed}
  GitAdd \defs GitAddOk \lor GitAddCannotAdd
\end{zed}

\subsection{Branch}

The branch command creates a new branch head with an associated name
which points to the current repository head. The operation only creates the new
branch and does not switch the working tree to it.

\begin{schema}{GitBranchOk}
	\Delta Repository \\
	name?: Name \\
	commit?: Commit \\
	response!: GitResponse
\where
	name? \notin \dom branch \\
	branch' = branch \cup \{name? \mapsto commit?\} \\
	head' = head \\
	index' = index \\
	workingCopy' = workingCopy \\
	tag' = tag \\
	parent' = parent \\
	response! = Ok
\end{schema}

Branch names are unique. If $name?$ is already assigned to a head, the operation
will fail:

\begin{schema}{GitBranchBranchAlreadyExists}
	\Xi Repository \\
	name?: Name \\
	commit?: Commit \\
	response!: GitResponse
\where
	name? \in \dom branch \\
  response! = BranchAlreadyExists
\end{schema}

We define \texttt{GitBranch} as the following disjunction:

\begin{zed}
	GitBranch \defs GitBranchOk \lor GitBranchBranchAlreadyExists
\end{zed}

\subsection{Tag}

Tag is one of the simplest Git operations. It labels a commit
with a name. One of the most popular use cases is for semantically
versioning projects.

\begin{schema}{GitTagOk}
	\Delta Repository \\
	name?: Name \\
	commit?: Commit \\
	response!: GitResponse
\where
	name? \notin \dom tag \\
	tag' = tag \cup \{name? \mapsto commit?\} \\
	head' = head \\
	index' = index \\
	workingCopy' = workingCopy \\
	branch' = branch \\
	parent' = parent \\
	response! = Ok
\end{schema}

As for branch naming, a commit cannot be tagged with the same value twice.
If $name?$ already exists, then the operation results in an error:

\begin{schema}{GitTagTagAlreadyExists}
	\Xi Repository \\
	name?: Name \\
	commit?: Commit \\
	response!: GitResponse
\where
	name? \in \dom tag \\
  response! = TagAlreadyExists
\end{schema}

We define \texttt{GitTag} as the following disjunction:

\begin{zed}
	GitTag \defs GitTagOk \lor GitTagTagAlreadyExists
\end{zed}

\subsection{Commit}

Commit stores the current contents of the index in a new commit along
with a message from the user describing the changes.

\begin{schema}{GitCommitOk}
	\Delta Repository \\
	author?: Author \\
	message?: Message \\
	response!: GitResponse
\where
	head.tree \neq index \\
	head'.author = author? \\
	head'.message = message? \\
	head'.tree = index \\
	parent' = parent \cup \{head' \mapsto \{head\} \} \\
	workingCopy' = index \\
	index' = index \\
	branch' = branch \\
	tag' = tag \\
	response! = Ok
\end{schema}

If the index tree has not been updated, the operation responds
with a \textit{nothing to commit} message:

\begin{schema}{GitCommitNothingToCommit}
	\Xi Repository \\
	author?: Author \\
	message?: Message \\
	response!: GitResponse
\where
	head.tree = index \\
	response! = NothingToCommit
\end{schema}

We define \texttt{GitCommit} as the following disjunction:

\begin{zed}
	GitCommit \defs GitCommitOk \lor GitCommitNothingToCommit
\end{zed}

\subsection{Reset Hard}

This reset form resets the index and working tree to a specific commit. Any
changes to known files in the working tree since the input commit are discarded.

\begin{schema}{GitResetHardOk}
  \Delta Repository \\
  commit?: Commit \\
  response!: GitResponse \\
  refs: \finset Commit
\where
  refs = \{ head \} \cup \ran branch \cup \ran tag \\
  \exists r: refs \spot commit? \in recursiveParents~(r,parent) \\
  head' = commit? \\
  index' = workingCopy' = commit?.tree \\
  branch' = branch \\
  tag' = tag \\
  parent' = parent \\
  response! = Ok
\end{schema}

If the $commit?$ does not belong to the repository, then the operation throws
an error:

\begin{schema}{GitResetHardCommitNotInRepository}
  \Xi Repository \\
  commit?: Commit \\
  response!: GitResponse \\
  refs: \finset Commit
\where
  refs = \{ head \} \cup \ran branch \cup \ran tag \\
  \lnot (\exists r: refs \spot commit? \in recursiveParents~(r,parent)) \\
  response! = CommitNotInRepository
\end{schema}

We define \texttt{GitResetHard} as the following disjunction:

\begin{zed}
  GitResetHard \defs GitResetHardOk \lor GitResetHardCommitNotInRepository
\end{zed}

