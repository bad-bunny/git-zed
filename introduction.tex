\section{Introduction}

Version control is a system that records changes to a file or set of files over
time so that you can recall specific versions later. Using a VCS generally means
that if you screw things up or lose files, you can easily recover with little
overhead. Git is a Distributed Version Control System (VCS). In a DVCS, clients
don't just check out the latest snapshot of the files; rather, they fully mirror
the repository, including its full history. Thus, if any server dies, any of the
client repositories can be copied back up to the server to restore it.

The major difference between Git and any other VCS is the way Git thinks about
its data. Conceptually, most othe systems store information as a list of
file-based changes. Git thinks of its data more like a series of snapshots of a
miniature filesystem. With Git, every time you commit, or save the state of your
project, Git basically takes a picture of what all your files look like at that
moment and stores a reference to that snapshot. To be efficient, if files have
not changed, Git doesn't store the file again, just a link to the previous
identical file it has already stored.

