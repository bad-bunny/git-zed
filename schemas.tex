\section{Schemas}

\subsection{FileMode}

\texttt{FileMode} represents the Unix file permissions for a file or directory.

\begin{schema}{FileMode}
  user: FilePermission \\
  group: FilePermission \\
  owner: FilePermission
\end{schema}

\subsection{Object}

\texttt{Object} is the smallest entry that can be tracked by a Git repository.

\begin{schema}{Object}
  mode: FileMode \\
	filename: Name
\end{schema}

\subsection{Tree}

Git relies on a tree data structure to track objects. This allows storing
individual files (blob) and groups of files (tree), similar to a UNIX
filesystem.

We define \texttt{Tree} as a schema with two constructors. First, we define
\texttt{treeBlob} as a constructor leaf nodes that map to files. Then, we use
\texttt{treeBranch} to construct non-leaf nodes that correspond to directories.

\begin{zed}
  Tree ::= treeBlob \ldata Object \rdata | treeBranch \ldata Object \cross \finset Tree \rdata
\end{zed}

The domain of \texttt{treeBlob} allows for the creation of a \texttt{Tree} given
any \texttt{Object}.  In contrast, \texttt{treeBranch} requires a parent
\texttt{Object} and a finite set of \texttt{Tree} elements that should be
considered its children.

\subsection{Commit}

A commit can be seen as a snapshot signed by an author along with a short
description.

\begin{schema}{Commit}
	author: Author \\
	message: Message \\
	tree: Tree
\end{schema}

