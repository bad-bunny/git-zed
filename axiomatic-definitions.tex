\section{Axiomatic Definitions}

\subsection{SubTrees}

We define \texttt{SubTrees} as the relation between a \texttt{Tree} $T$ and the
finite set of all \texttt{Tree} elements that can be reached by traversing $T$
(including $T$).

\begin{axdef}
  subTrees: Tree \fun \finset Tree
\where
  \exists a,b,c: Tree; bo: Object; bc: \finset Tree \spot a \in subTrees~b \iff \\
  \t1 a = b \lor \\
  \t1 b \in \ran treeBranch \land b = treeBranch~(bo,bc) \land c \in bc \land a \in subTrees~c
\end{axdef}

To check if $a$ is a sub-tree of $b$, we evaluate two possible scenarios that
would make this expression true:

\begin{enumerate}
  \item $a = b$
  \item $b$ is a \texttt{treeBranch} and $a$ is a sub-tree of at least of $b$'s
  children ($bc$).
\end{enumerate}

\subsection{TreeWithFile}

One of the most common tasks for a Git repository is adding or removing nodes
from an  existing tree; these nodes can be either files or directories. For this
task we define the partial function \texttt{treeWithFile}. The domain of
\texttt{treeWithFile} contains tuples of the form $(src, path, node)$, where
$src$ corresponds to the tree that will be modified, $node$ is the element that
will be added and $path$ is the insertion point for $node$.  The range defines
the resulting \texttt{Tree}.

\begin{axdef}
  treeWithFile: Tree \cross Tree \cross Tree \pfun Tree
\where
  \exists src, path, file, new, c: Tree; so: Object; sc, sc': \finset Tree \spot new = treeWithFile~(src, path, file) \iff \\
  \t1 src = treeBranch~(so,sc) \land \\
  \t1 path \in \ran treeBranch \land \\
  \t1 path \in subTrees~src \land \\
    \t2 ( src = path \land new = treeBranch~(so, sc \cup \{ file \}) \lor \\
    \t2 sc' = \{ c: sc | path \notin subTrees~c \} \cup \\
      \t3 \{ t: Tree | \forall c: sc \spot path \in subTrees~c \land t = treeWithFile~(c, path, file) \} \land \\
      \t3 new = treeBranch~(so, sc') )
\end{axdef}

We first define the following preconditions:

\begin{enumerate}
  \item $src$ is not a leaf node.
  \item $path$ is not a leaf node.
  \item $path$ is a sub-tree of $src$.
\end{enumerate}

Next we define two possible cases:

\begin{enumerate}
  \item $src$ is the insertion point and $node$ is added to $src$'s children.
  \item $node$ is added (recursively using \texttt{treeWithFile}) to every child
  of $src$ that contains $path$.
\end{enumerate}

\subsection{Hash}

\texttt{HashTree} is a surjective function which associates a \texttt{Tree} to a
hash value.

\begin{axdef}
  hashTree: Tree \surj Digest
\end{axdef}

Likewise, \texttt{HashCommit} is a surjective function which associates a
\texttt{Commit} to a hash value.

\begin{axdef}
  hashCommit: Commit \surj Digest
\end{axdef}

\subsection{RecursiveParents}

The relation \texttt{RecursiveParents} allows us to find all the ancestors of a given
\texttt{Commit} in a repository. The domain includes the reference commit and the parent
relation of the repository and the range includes the finite set of parent commits.

\begin{axdef}
  recursiveParents: Commit \cross (Commit \pfun \power Commit) \pfun \finset Commit
\where
  \exists ref: Commit; parent: Commit \pfun \power Commit; result: \finset Commit \spot \\
  \t1 result = recursiveParents~(ref, parent) \iff \\
    \t2 ref \in \dom parent \land \\
    \t2 (\forall c: result \spot c = ref \lor (\exists p: parent~ref \spot c \in recursiveParents~(p, parent)))
\end{axdef}

The only precondition for this relation is that the reference $Commit$ is in the
repository; this is evaluated by checking if the reference $Commit$ is in the
domain of the repository's parent relation. Given this restriction, we can
define the recursive parents for a $Commit$ as itself, its parents and the
recursive parents for each of its parents.
