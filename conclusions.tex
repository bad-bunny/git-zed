\section{Conclusions}

To the best of our knowledge, our specification of the Git version control
system adheres very closely to the actual implementation. We aligned our schema
definitions to Git's object and tree structures. Also, we operated over the
working copy, index and commit trees in a way that mimics, or at least closely
resembles, Git's actual behavior.

The translation of recursive constructs into Z was not intuitive; however, we
found that defining a tree structure using two different constructors for
branches and leaf nodes allowed us to express recursive structures in a way that
was both simple and easy to understand. Nevertheless, these constructors by
themselves were not enough to define recursive operations. We attempted to solve
this issue with traversal sets and existence constraints, but the best solution
for our use case ended up being recursive axiomatic definitions. We also found
that the upwards traversal of a tree to recalculate digests was too complex and
did not provide much in terms of specification constraints. Instead, we decided
to follow the example of Morgan and Sufrin~\cite{morgan} by defining global
relations to reference the hash digests of both objects and commits.

